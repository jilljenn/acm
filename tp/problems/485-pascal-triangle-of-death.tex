% http://uva.onlinejudge.org/external/4/485.html

\problem{Pascal's Triangle of Death}

\noindent
UVA \no 485\bigskip

In this problem, you are asked to generate Pascal's Triangle. Pascal's Triangle is useful in many
areas from probability to polynomials to programming contests. It is a triangle of integers with
``1'' on top and down the sides. Any number in the interior equals the sum of the two numbers
above it. For example, here are the first 5 rows of the triangle.

\begin{verbatim}
1  
1 1  
1 2 1  
1 3 3 1  
1 4 6 4 1
\end{verbatim}

In ``Pascal's Triangle of Death,'' you are to generate a left justified Pascal's Triangle. When any
number in the triangle is exceeds or equals 1060, your program should finish printing the
current row and exit. The output should have each row of the triangle on a separate line with one
space between each element.

The final element of each line should be directly followed by a
newline. There is no space after the last number on each line.

\subsection*{Sample Input}

There is no input for this problem.

\subsection*{Sample Output}

\begin{verbatim}
1
1 1
1 2 1
1 3 3 1
1 4 6 4 1
.
.
.
\end{verbatim}

etc.
