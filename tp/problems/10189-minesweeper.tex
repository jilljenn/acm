% http://uva.onlinejudge.org/external/101/10189.html
 
\problem{Minesweeper}

\subsection*{The Problem}

Have you ever played Minesweeper? It's a cute little game which comes within a certain 
Operating System which name we can't really remember. Well, the goal of the game is to
find where are all the mines within a MxN field. To help you, the game shows a number
in a square which tells you how many mines there are adjacent to that square. For instance,
supose the following 4x4 field with 2 mines (which are represented by an * character):

\begin{verbatim}
*...
....
.*..
....
\end{verbatim}

If we would represent the same field placing the hint numbers described above, we would end up with:

\begin{verbatim}
*100
2210
1*10
1110
\end{verbatim}

As you may have already noticed, each square may have at most 8 adjacent squares.

\subsection*{The Input}

The input will consist of an arbitrary number of fields. The first line of each field contains
two integers $n$ and $m$ ($0 < n,m \leqslant 100$) which stands for the number of lines and columns of the field
respectively. The next n lines contains exactly m characters and represent the field. Each safe
square is represented by an "." character (without the quotes) and each mine square is represented
by an "*" character (also without the quotes). The first field line where n = m = 0 represents the end 
of input and should not be processed.

\subsection*{The Output}

For each field, you must print the following message in a line alone:

\begin{verbatim}
Field #x:
\end{verbatim}

Where x stands for the number of the field (starting from 1). The next n lines should
contain the field with the "." characters replaced by the number of adjacent mines
to that square. There must be an empty line between field outputs.

\subsection*{Sample Input}

\begin{verbatim}
4 4
*...
....
.*..
....
3 5
**...
.....
.*...
0 0
\end{verbatim}

\subsection*{Sample Output}

\begin{verbatim}
Field #1:
*100
2210
1*10
1110

Field #2:
**100
33200
1*100
\end{verbatim}

%© 2001 Universidade do Brasil (UFRJ). Internal Contest Warmup 2001.
