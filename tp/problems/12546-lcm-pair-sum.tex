% http://uva.onlinejudge.org/index.php?option=com_onlinejudge&Itemid=8&category=441&page=show_problem&problem=3991

\problem{LCM Pair Sum}

\noindent
UVa 12546 (SWERC 2012)\bigskip

One of your friends desperately needs your help. He is working with a secret agency and doing some encoding stuffs. As the mission is confidential he does not tell you much about that, he just want you to help him with a special property of a number. This property can be expressed as a function $f(n)$ for a positive integer $n$. It is defined as:

\[ f(n) = \sum_{\substack{1 \leqslant p \leqslant q \leqslant n\\lcm(p, q) = n}} (p + q) \]

In other words, he needs the sum of all possible pairs whose least common multiple is $n$. (The least common multiple (LCM) of two numbers $p$ and $q$ is the lowest positive integer which can be perfectly divided by both $p$ and $q$). For example, there are 5 different pairs having their LCM equal to 6 as $(1, 6), (2, 6), (2, 3), (3, 6), (6, 6)$. So $f(6)$ is calculated as $f(6) = (1+6)+(2+6)+(2+3)+(3+6)+(6+6) = 7 + 8 + 5 + 9 + 12 = 41$.
Your friend knows you are good at solving this kind of problems, so he asked you to lend a hand. He also does not want to disturb you much, so to assist you he has factorized the number. He thinks it may help you.

\subsection*{Input}

The first line of input will contain the number of test cases $T$ ($T \leqslant 500$). After that there will be $T$ test cases. Each of the test cases will start with a positive number $C$ ($C \leqslant 15$) denoting the number of prime factors of $n$. Then there will be $C$ lines each containing two numbers $P_i$ and $a_i$ denoting the prime factor and its power ($P_i$ is a prime between 2 and 1000) and ($1 \leqslant a_i \leqslant 50$). All the primes for an input case will be distinct.

\subsection*{Output}

For each of the test cases produce one line of output denoting the case number and $f(n)$ modulo 1000000007. See the output for sample input for exact formatting.

\subsection*{Sample Input 1}

\begin{verbatim}
3
2
2 1
3 1
2
2 2
3 1
1
5 1
\end{verbatim}

\subsection*{Sample Output}

\begin{verbatim}
Case 1: 41
Case 2: 117
Case 3: 16
\end{verbatim}
